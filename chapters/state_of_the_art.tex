\chapter{State of the Art}
We will introduce in this chapter some useful definitions and concepts that she
light on  the project. We will also assess the current situation be examining
the literature as well as the current solutions adopted by Predictix. Next, we
will specify the different proposed solutions and choose the one that fits most
our needs.



\subsection{General concepts}

In this section, we will be defining the general concepts that the Technical
Infrastructure team work around, and more specifically those that this project
evolves around. We will also clarify the use cases for these technologies and
concepts in Predictix.

\subsubsection{Declarative programming}
Definition:
Declarative programming is a programming paradigm where the developer write code
in such a way that it describes the logic of computation, without describing its
workflow.

\subsubsection{REST architecture}
Rest stands fore REpresntational State Transfer. It is an architecture style for
designing networked applications. It permits creating, modifying resources
easily. Indeed REST is a lightweight alternative to complex mechanism like RPC,
CORBA and SOAP. 

Rest is not a 'standard'. In fact,  it is a guideline to build an efficient
framework for communication between two machines using HTTP protocol. The World
Wide Web itself, based on HTTP, can be viewed as a REST-based architecture. REST
relies on a stateless, client-server, cacheable communication protocol. It is
simple to implement and maintain. In addition, it allows applications to be scalable
by supporting multiple backend services at the same time.

Much like Web Services, a REST service is platform-independent,
language-independent standard-based asit runs on top of HTTP, and is easily used
in the presence of firewalls.

However, there are a few major concepts which make REST unnique from other web
services. In fact , its main key principales are the following:

Unique URL-Resource mapping: Every resouce is mapped to a unique URL. That
refers to a some logical way  to access information. 

Statelessness: All information required to precess the request by server is
contained along with the request. This means that no informatio of the previes
request is maintained by the server. This is inherited from the fact that REST
is based on HTTP.

Action Verbs: REST architecture use HTTP verbs to identify the apropriate
action. The main HTTP verbs used in a REST architecture are GET, POST, PUT and
DELETE. In fact, GET is used by the client to access the resource on the server,
PUT to update a resource, POST to create a new one and DELETE to remove
resource.

Data Exchange formats: REST architecture does not require any particular
encoding for the resource body. JSON and XML are the most used format, but it
can be PROTOBUF, YAML etc. 

\subsection{Cloud}

Cloud computing is a type of Internet-based computing that provides shared
computer processing resources and data to computers and other devices on demand.
It is a model for enabling ubiquitous, on-demand access to a shared pool of
configurable computing resources (e.g., computer networks, servers, storage,
applications and services), which can be rapidly provisioned and released
with minimal management effort. Cloud computing and storage solutions provide
users and enterprises with various capabilities to store and process their data
in either privately owned, or third-party data centers[3] that may be located
far from the user–ranging in distance from across a city to across the world.
Cloud computing relies on sharing of resources to achieve coherence and economy
of scale, similar to a utility (like the electricity grid) over an electricity
network.

\section{Definitions}

\subsubsection{AWS}
Amazon Web Services (AWS) is a subsidiary of Amazon.com that provides on-demand
cloud computing platforms. These services operate from many global geographical
regions including 6 in North America. They include Amazon Elastic Compute
Cloud, also known as "EC2", and Amazon Simple Storage Service, also known as
"S3". As of 2016, AWS has more than 70 services, spanning a wide range,
including compute, storage, networking, database, analytics, application
services, deployment, management, mobile, developer tools and tools for the
Internet of Things. Amazon markets AWS as a service to provide large computing
capacity quicker and cheaper than a client company building an actual physical
server farm.

\subsection{Retail}
TODO

\subsection{SaaS}
The disruption to the widespread on-premise and self-hosted retail forecasting
and TODO applications begin with the emerge of the cloud.
TODO

\subsection{Benchmark}
TODO

\subsection{Nix}

Nix is a purely functional package manager. This means that it treats packages
like values in purely functional programming languages such as Haskell — they
are built by functions that don’t have side-effects, and they never change after
they have been built. Nix stores packages in the Nix store, usually the
directory /nix/store, where each package has its own unique subdirectory such as

\colorbox{Gray}{\lstinline{nix/store/b6gvzjyb2pg0kjfwrjmg1vfhh54ad73z-firefox-33.1/}}

Where \colorbox{Gray}{\lstinline{b6gvzjyb2pg0...}} is a unique identifier for the package that captures all its
dependencies (it’s a cryptographic hash of the package’s build dependency
graph). This enables many powerful features.

Nix was created by Eelco Dolstra as he's PHD. Eelco is, along with other main
contributors to the Nix ecosystem, a Predictix employment.
\subsection{Nix language}


\subsubsection{NixOS}
Nixos is Linux distribution built on top of the Nix package manager. It
uses the Nix language for declarative configuration which allows realiable
system upgrades.

\subsubsection{NixOps}
NixOps is a tool for deploying sets of NixOS Linux Machines, either to real
hardware or to virtual machines. It extends NixOS's declarative approach to
system configuration to networks and adds provisioning.

\par
The NixOps Dashboard is an extension to NixOps and offers the same content and
functionality via a Web UI and a rich RESTful API. Core features are:

\begin{itemize}
\item User-friendly web-based interface
\item Improved security
\item Scheduled operations (deployments, backups, ...etc)
\item Permanent traceability
\item Clear audit trail
\item Role based access
\end{itemize}

